\documentclass[12pt,a4paper]{book}

% 使用Package
\usepackage{fontspec}
\usepackage{titlesec}
\usepackage{titletoc}
\usepackage{xCJKnumb}

% Package設定
\setromanfont{Noto Sans CJK TC Regular} % 儷黑Pro
\setmonofont{Courier New} % 等寬字型
\XeTeXlinebreaklocale zh
\XeTeXlinebreakskip = 0pt plus 1pt
\setcounter{chapter}{-1}

% 標題
\title{C語言,從崩潰到放棄}

% 正文開始
\begin{document}

\chapter{開卷}
\begin{block}{bg=grey}
    凡人若想要勝過天才,就只能化身為修羅。
\end{block}

當你打開了這本講義,恭喜你成為了藝術家!或許你會問:「為什麼是藝術家?不是工程師嗎?」原因很簡單,因為「Coding也是門藝術」,或許會有人覺得相當不服氣,「不就是寫幾行程式碼而已嗎?」正是因為能用幾行程式碼改變世界,才會稱他為「藝術」。在撰寫程式碼上,有許許多多需要瞭解、注意的事情。

在這本講義中,集結了編者群的學習歷程,每個人都是從基礎一步一步走來,正如同翻閱這份講義的各位,我們期望讓這份講義,不只是冷冰冰的白紙黑字,理論與實作並進,透過課堂上的範例及課後的練習,堆起名為實力的巨牆,奠定資訊工程的基礎。


\chapter{魔工鑄器 Environment}
\section{Dev-C++}
\section{Code::Blocks}
\section{CLion}
\section{Visual Studio}

\chapter{鑑往知來 C Intro}

\chapter{五型殺氣 Basic I/O}

\chapter{扮裝行列 Operation}

\chapter{循環演算 Loop \& Array}

\chapter{情報強化 String}

\chapter{三千世界 Function}

\chapter{鬼門遁甲 Pointer}

\chapter{七夜怪談 Structure}

\chapter{剪紙成兵 File I/O}



\end{document}
